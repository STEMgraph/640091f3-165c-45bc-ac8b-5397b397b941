\learningobjective{At the end of this challenge, the scholar will be able to use shell functions in their own sessions and understand the concept of running them in encapsulated environments.}

\begin{challenge}
    \chatitle{Running Shell Functions in Encapsulated Environments} 

    \begin{chadescription}
        Self-defined shell functions are a way to extend the shell's functionality and make repeatable tasks easier to perform.
        Memorizing them in your current terminal session comes with some drawbacks, that already early programmers had to face.
        Next to some minor issues, there is a major one: functions and identifiers, that get sourced in a terminal session may overwrite earlier defined functions or identifiers.
        This can lead to unexpected behavior.
        Especially when the code is ported to a different machine.
        Don't understand this as \textit{Do never source functions or identifiers}.
        There are definitly reasons why one would do that, sometimes even to overwrite some already defined identifiers purposefully.
        But in general, we would much rather like to run functions in their own environment.
    \end{chadescription}

    \begin{task}
        We start by writing a function and storing it in a file.
        \begin{lstlisting}
            echo 'function multiply() {
            echo $(( $1 * $2 ))
            }' > multiply.sh
        \end{lstlisting}
        
        \begin{enumerate}
            \item Write down a skill, on a piece of paper that you really like.
            \item Come up with a good explaination, why this skill is useful in STEM. 
            \item Explain, why this is a basic task to perform.
            \item Think of 3 to 4 applications of this skill.
            \item Answer the questions below.
            \item Write a request to ChatGPT using the following prompt:
            \begin{lstlisting}
                I am working on github.com/STEMgraph challenges.
                Today I worked on the following:
                ```
                <insert raw text of this challenge here>
                ```
                Please give me detailed feedback to my answers to the provided questions.
                Here are my answers in the correct order:
                ```
                <insert your answers to the tasks questions here>
                ```
            \end{lstlisting}
        \end{enumerate}

%% The questions should be structured carefully and don't ask to much information at once. 
%% Especially when a Chatbot should be able to correct the scholars answers, the questions should be precise and concise.
%% Have at least three questions to every task.
        \begin{questions}
            \item What is the skill you want to teach your scholars?
            \item Why do you think its a foundationary skill?
            \item What are the applications of this skill?
            \item What prerequisites do you need to perform this skill?
            \item What is the first step to perform this skill?
            \item What advice do you have for the scholar at the end of this challenge?
        \end{questions}
    \end{task}

%% If you want the scholar to take some further learning from the tasks that they just performed, use the advice tag.
%% The advise should not be longer than maybe 80 words.
    \begin{advice}
        Let someone from your peer-group try out your challenges!
        Have them generate issues in the github repository and let them write down how long it took them to complete the challenge.
        A good challenge is one, that takes less than 30 minutes to complete.
        This way, scholars can plan their day.
        Happy generating!
    \end{advice}
\end{challenge}
